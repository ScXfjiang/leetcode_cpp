\chapter{Queue}
\section{LC 0232 - Implement Queue using Stacks}
Implement a \ul{first in first out (FIFO) queue} using only \ul{two stacks}.\\

Implement the {\colorbox{CodeBackground}{\lstinline|MyQueue|}} class:
\begin{itemize}
\item {\colorbox{CodeBackground}{\lstinline|void push(int x)|}} - Pushes element x to the back of the queue.
\item {\colorbox{CodeBackground}{\lstinline|int pop()|}} - Removes the element from the front of the queue and returns it.
\item{\colorbox{CodeBackground}{\lstinline| int peek()|}} ({\colorbox{CodeBackground}{\lstinline|std::queue::front()|}}) - Returns the element at the front of the queue.
\item {\colorbox{CodeBackground}{\lstinline|bool empty()|}} - Returns {\colorbox{CodeBackground}{\lstinline|true|}} if the queue is empty, {\colorbox{CodeBackground}{\lstinline|false|}} otherwise.
\end{itemize}

\subsection*{Solution}
\begin{lstlisting}
class MyQueue {
 public:
  MyQueue() {}

  void push(int x) { in_stk.push(x); }

  int pop() {
    if (out_stk.empty()) { transfer(); }
    int front = out_stk.top();
    out_stk.pop();
    return front;
  }

  int peek() {  // std::queue::front()
    if (out_stk.empty()) { transfer(); }
    return out_stk.top();
  }

  bool empty() { return in_stk.empty() && out_stk.empty(); }

 private:
  std::stack<int> in_stk;
  std::stack<int> out_stk;

  // transfer elements from in_stk to out_stk
  void transfer() {
    while (!in_stk.empty()) {
      out_stk.push(in_stk.top());
      in_stk.pop();
    }
  }
};
\end{lstlisting}

\section{LC 0225 - Implement Stack using Queues}
Implement a \ul{last-in-first-out (LIFO) stack} using only \ul{two queues}.\\

Implement the {\colorbox{CodeBackground}{\lstinline|MyStack|}} class:
\begin{itemize}
\item {\colorbox{CodeBackground}{\lstinline|void push(int x)|}} - Pushes element {\colorbox{CodeBackground}{\lstinline|x|}} to the top of the stack.
\item {\colorbox{CodeBackground}{\lstinline|int pop()|}} - Removes the element on the top of the stack and returns it.
\item {\colorbox{CodeBackground}{\lstinline|int top()|}} - Returns the element on the top of the stack.
\item {\colorbox{CodeBackground}{\lstinline|bool empty()|}} - Returns {\colorbox{CodeBackground}{\lstinline|true|}} if the stack is empty, {\colorbox{CodeBackground}{\lstinline|false|}} otherwise.
\end{itemize}

\subsection*{Solution 1 - Two Queues}
\begin{lstlisting}
class MyStack {
 public:
  MyStack() {}

  void push(int x) {
    q2.push(x);
    while (!q1.empty()) {
      q2.push(q1.front());
      q1.pop();
    }
    std::swap(q1, q2);
  }

  int pop() {
    if (q1.empty()) { return -1; }
    int top = q1.front();
    q1.pop();
    return top;
  }

  int top() {
    if (q1.empty()) { return -1; }
    return q1.front();
  }

  bool empty() { return q1.empty(); }

 private:
  std::queue<int> q1;
  std::queue<int> q2;
};
\end{lstlisting}

\subsection*{Solution 2 - One Queue}
\begin{lstlisting}
class MyStack {
 public:
  void push(int x) {
    q.push(x);
    for (int i = 0; i < q.size() - 1; ++i) {
      q.push(q.front());
      q.pop();
    }
  }

  int pop() {
    int top = q.front();
    q.pop();
    return top;
  }

  int top() { return q.front(); }

  bool empty() { return q.empty(); }

 private:
  std::queue<int> q;
};
\end{lstlisting}

\chapter{Priority Queue}
\section{LC 0347 - Top K Frequent Elements}
Given a \ul{non-empty} integer array {\colorbox{CodeBackground}{\lstinline|nums|}} and an integer {\colorbox{CodeBackground}{\lstinline|k|}} ({\colorbox{CodeBackground}{\lstinline|k >= 1|}}), return the {\colorbox{CodeBackground}{\lstinline|k|}} most frequent elements. You may return the answer in any order.\\

Examples:
\begin{itemize}
\item {\colorbox{CodeBackground}{\lstinline|nums = [1,1,1,2,2,3], k = 2 --> [1,2]|}}
\item {\colorbox{CodeBackground}{\lstinline|nums = [1], k = 1 --> [1]|}}
\end{itemize}

\subsection*{Solution - Priority Queue}
\begin{lstlisting}
std::vector<int> topKFrequent(std::vector<int>& nums, int k) {
  std::unordered_map<int, int> num2freq;
  for (int num : nums) { num2freq[num]++; }
  auto comp = [&num2freq](int n1, int n2) { return num2freq[n1] > num2freq[n2]; };
  std::priority_queue<int, std::vector<int>, decltype(comp)> heap(comp);
  for (auto& [num, freq] : num2freq) {
    heap.push(num);
    if (heap.size() > k) { heap.pop(); }
  }
  std::vector<int> top_k;
  while (!heap.empty()) {
    top_k.push_back(heap.top());
    heap.pop();
  }
  return top_k;
}
\end{lstlisting}

\section{LC 0148 - Sort List}
Given the {\colorbox{CodeBackground}{\lstinline|head|}} of a \ul{singly linked list}, return the list after sorting it in ascending order.\\

Examples:
\begin{itemize}
\item {\colorbox{CodeBackground}{\lstinline|[4 -> 2 -> 1 -> 3] --> [1 -> 2 -> 3 -> 4]|}}
\item {\colorbox{CodeBackground}{\lstinline|[-1 -> 5 -> 3 -> 4 -> 0] --> [-1 -> 0 -> 3 -> 4 -> 5]|}}
\item {\colorbox{CodeBackground}{\lstinline|[] --> []|}}
\end{itemize}

\subsection*{Solution - Priority Queue}\label{solution:lc-148_priority_queue}
\begin{lstlisting}
ListNode* sortList(ListNode* head) {
  if (!head) { return nullptr; }
  auto compare = [](const ListNode* lhs, const ListNode* rhs) {
    return lhs->val > rhs->val;
  };
  std::priority_queue<ListNode*, std::vector<ListNode*>, decltype(compare)> min_heap(
      compare);
  while (head) {
    min_heap.push(head);
    head = head->next;
  }
  ListNode* dummy_head = new ListNode();
  ListNode* curr = dummy_head;
  while (!min_heap.empty()) {
    curr->next = min_heap.top();
    min_heap.pop();
    curr = curr->next;
  }
  curr->next = nullptr;
  return dummy_head->next;
}
\end{lstlisting}

\subsection*{Other Solutions}
\begin{itemize}
\item \hyperref[solution:lc0148_merge_sort]{Merge Sort}
%\item \hyperref[solution:lc-148_priority_queue]{Priority Queue}
\end{itemize}