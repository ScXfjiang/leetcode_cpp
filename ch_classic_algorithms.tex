\chapter{Classic Algorithms}\label{ch:classic_algorithms}
\section{LC 0169 - Majority Element}
Given an array {\colorbox{CodeBackground}{\lstinline|nums|}} of size {\colorbox{CodeBackground}{\lstinline|n|}}, return the \ul{majority element}.\\

The majority element is the element that appears more than {\colorbox{CodeBackground}{\lstinline|n / 2|}} times. You may assume that the majority element always exists in the array.\\

Examples:
\begin{itemize}
	\item {\colorbox{CodeBackground}{\lstinline|nums = [3,2,3] --> 3|}}
	\item {\colorbox{CodeBackground}{\lstinline|nums = [2,2,1,1,1,2,2] --> 2|}}
\end{itemize}

\subsection*{Solution - Boyer–Moore Majority Vote Algorithm}
\begin{lstlisting}
int majorityElement(std::vector<int>& nums) {
	int candidate = 0;
	int count = 0;
	for (auto num : nums) {
		if (count == 0) { candidate = num; }
		count += (num == candidate) ? 1 : -1;
	}
	return candidate;
}
\end{lstlisting}