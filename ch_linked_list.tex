\chapter{Linked List}
\section{Thinking in Singly Linked Lists}
\subsection{Head and Tail Pointers}
In a {\color{blue}{singly linked list}}, the {\color{blue}{head}} pointer points to the {\color{blue}{first node}}, and the {\color{blue}{tail}} pointer points to the {\color{blue}{last node}}. \\

{\color{blue}{A singly linked list is empty}} $\Leftrightarrow$ {\colorbox{CodeBackground}{\lstinline|head == nullptr|}} and {\colorbox{CodeBackground}{\lstinline|tail == nullptr|}}.

\subsection{Dummy Head}
In scenarios where the {\color{blue}{first node}} requires different handling than the rest (e.g., {\color{blue}{insertion}}, {\color{blue}{deletion}}, some {\color{blue}{rearranging operations}}), a {\color{blue}{dummy head}} helps in treating all nodes uniformly.\\

{\color{blue}{A singly linked list with the dummy head is empty}} $\Leftrightarrow$ {\colorbox{CodeBackground}{\lstinline|dummy_head = new Node();|}} and {\colorbox{CodeBackground}{\lstinline|tail == nullptr|}}.

\subsection{Practical Tips}
\begin{itemize}
	\item To {\color{blue}{insert}} or {\color{blue}{remove}} a node from a linked list, you first need to {\color{blue}{locate the node immediately preceding the target node}}.
	\item The meaning of the {\colorbox{CodeBackground}{\lstinline|curr|}} node varies with the context. During {\color{blue}{traversal}}, {\colorbox{CodeBackground}{\lstinline|curr|}} refers to the {\color{blue}{node currently under examination}}. However, in operations like {\color{blue}{insertion}} or {\color{blue}{deletion}}, {\colorbox{CodeBackground}{\lstinline|curr|}} refers to the {\color{blue}{node immediately preceding the target node}}, which is also the node whose {\colorbox{CodeBackground}{\lstinline|next|}} pointer is going to be modified.
\end{itemize}

\subsection{Code Snippet}
\hdashrule[0.5ex]{\linewidth}{0.5pt}{1mm 3pt}
Unless specified, {\color{blue}{nodes}} of {\color{blue}{singly linked list}} are defined as follows:
\begin{lstlisting}
struct ListNode {
	ListNode() : val(0), next(nullptr) {}
	ListNode(int x) : val(x), next(nullptr) {}
	ListNode(int x, ListNode* next) : val(x), next(next) {}
	
	int val;
	ListNode* next;
};
\end{lstlisting}
\hdashrule[0.5ex]{\linewidth}{0.5pt}{1mm 3pt}

\subsubsection*{Insert {\colorbox{CodeBackground}{\lstinline|y|}} After {\colorbox{CodeBackground}{\lstinline|x|}}: {\colorbox{CodeBackground}{\lstinline|[x -> z] --> [x -> y -> z]|}}}
\begin{lstlisting}
void insert(ListNode* x, ListNode* y) {
	y->next = x->next;
	x->next = y;
}
\end{lstlisting}

\subsubsection*{Delete {\colorbox{CodeBackground}{\lstinline|y|}} After {\colorbox{CodeBackground}{\lstinline|x|}}: {\colorbox{CodeBackground}{\lstinline|[x -> y -> z] --> [x -> z]|}}}
\begin{lstlisting}
void delete_node(ListNode* x, ListNode* y) {
	x->next = y->next;
	delete y;
}
\end{lstlisting}

\subsubsection*{Traverse the List}
\begin{lstlisting}
void traverse(ListNode* head) {
	ListNode* curr = head;
	while (curr) {
		// do something
		curr = curr->next;
	}
}
\end{lstlisting}

\subsubsection*{Search For a Node}
\begin{lstlisting}
ListNode* find(ListNode* head, bool (*condition)(ListNode*)) {
	ListNode* curr = head;
	while (curr && !condition(curr)) { curr = curr->next; }
	return curr;
}
\end{lstlisting}

\section{Thinking in Doubly Linked Lists}
TODO

\section{LC 0141 - Linked List Cycle}\label{lc0141}
Given the {\colorbox{CodeBackground}{\lstinline|head|}} of a \ul{singly linked list}, determine if the singly linked list has a \ul{cycle} in it.

\subsection*{Solution 1 - Tortoise and Hare Algorithm}
\begin{lstlisting}
bool hasCycle(ListNode *head) {
	if (!head) { return false; }
	ListNode *slow = head;
	ListNode *fast = head->next;
	while (fast && fast->next) {
		if (slow == fast) { return true; }
		slow = slow->next;
		fast = fast->next->next;
	}
	return false;
}
\end{lstlisting}

\subsection*{*Solution 2 - Hash Set (Extra Space)}
\begin{lstlisting}
bool hasCycle(ListNode* head) {
	std::unordered_set<ListNode*> visited;
	ListNode* curr = head;
	while (curr) {
		if (visited.find(curr) != visited.end()) { return true; }
		visited.insert(curr);
		curr = curr->next;
	}
	return false;
}
\end{lstlisting}

\section{LC 0206 - Reverse Linked List}
Given the {\colorbox{CodeBackground}{\lstinline|head|}} of a \ul{singly linked list}, reverse the list, and return the reversed list.\\

Examples:
\begin{itemize}
	\item {\colorbox{CodeBackground}{\lstinline|[1 -> 2 -> 3 -> 4 -> 5] --> [5 -> 4 -> 3 -> 2 -> 1]|}}
\end{itemize}

\subsection*{Solution 1 - Iterative}\label{solution:lc0206_iterative1}
Reverse the direction of the pointers in the list.
\begin{lstlisting}
ListNode* reverseList(ListNode* head) {
	ListNode* prev = nullptr;
	ListNode* curr = head;
	ListNode* next = nullptr;
	while (curr) {
		next = curr->next;
		curr->next = prev;
		prev = curr;
		curr = next;
	}
	return prev;
}
\end{lstlisting}

\subsection*{*Solution 2 - Iterative (Tricky)}\label{solution:lc0206_iterative2}
Iteratively insert each node to the front of the reversed list.
\begin{lstlisting}
ListNode* reverseList(ListNode* head) {
	ListNode* dummy_head = new ListNode();
	ListNode* curr = head;
	while (curr) {
		ListNode* next = curr->next;
		curr->next = dummy_head->next;
		dummy_head->next = curr;
		curr = next;
	}
	return dummy_head->next;
}
\end{lstlisting}

\subsection*{Other Solutions}
\begin{itemize}
	\item \hyperref[solution:lc0206_recursion]{Recursion}
\end{itemize}

\section{LC 0092 - Reverse Linked List II}
Given the {\colorbox{CodeBackground}{\lstinline|head|}} of a \ul{singly linked list} and two integers {\colorbox{CodeBackground}{\lstinline|left|}} and {\colorbox{CodeBackground}{\lstinline|right|}} where {\colorbox{CodeBackground}{\lstinline|left <= right|}}, reverse the nodes of the list from \ul{position {\colorbox{CodeBackground}{\lstinline|left|}}} to \ul{position {\colorbox{CodeBackground}{\lstinline|right|}}}, and return the reversed list.

\subsection*{Solution 1}
Reverse the direction of the pointers in the sublist.
\begin{lstlisting}
ListNode* reverseBetween(ListNode* head, int left, int right) {
	if (!head || left == right) { return head; }
	ListNode* dummy_head = new ListNode();
	dummy_head->next = head;
	ListNode* prev = dummy_head;
	// move prev to the node just before the left node
	for (int i = 0; i < left - 1; ++i) { prev = prev->next; }
	// reverse the sublist from left to right
	ListNode* curr = prev->next;
	ListNode* next = nullptr;
	for (int i = 0; i < right - left; ++i) {
		next = curr->next;
		curr->next = next->next;
		next->next = prev->next;
		prev->next = next;
	}
	return dummy_head->next;
}
\end{lstlisting}

\subsection*{Solution 2}
Iteratively insert each node to the front of the reversed sublist.
\begin{lstlisting}
ListNode* reverseBetween(ListNode* head, int left, int right) {
	if (!head || left == right) { return head; }
	ListNode* dummy_head = new ListNode();
	dummy_head->next = head;
	ListNode* prev = dummy_head;
	// move prev to the node before the left position
	for (int i = 0; i < left - 1; ++i) prev = prev->next;
	// iteratively insert the node to the front of the reversed sublist
	ListNode* start = prev->next;
	ListNode* then = start->next;
	for (int i = 0; i < right - left; ++i) {
		start->next = then->next;
		then->next = prev->next;
		prev->next = then;
		then = start->next;
	}
	return dummy_head->next;
}
\end{lstlisting}

\section{LC 0061 - Rotate List}\label{lc0061}
Given the {\colorbox{CodeBackground}{\lstinline|head|}} of a \ul{singly linked list}, \ul{rotate the list to the right by {\colorbox{CodeBackground}{\lstinline|k|}} steps}, where {\colorbox{CodeBackground}{\lstinline|k|}} is non-negative.

\subsection*{Solution}
\begin{lstlisting}
ListNode* rotateRight(ListNode* head, int k) {
	if (head == nullptr || k == 0) { return head; }
	// determine len of linked list
	ListNode* curr = head;
	int len = 1;
	while (curr->next) {
		++len;
		curr = curr->next;
	}
	// connect tail to head
	curr->next = head;
	// find new tail
	int steps_to_new_tail = len - k % len;
	while (steps_to_new_tail--) { curr = curr->next; }
	// break circle
	head = curr->next;
	curr->next = nullptr;
	return head;
}
\end{lstlisting}

\subsection*{Related}
\begin{itemize}
	\item \hyperref[lc0189]{LC 0189 - Rotate Array}
	\item \hyperref[lc0061]{LC 0061 - Rotate List}
\end{itemize}

\section{LC 0086 - Partition List}\label{lc0086}
Given the {\colorbox{CodeBackground}{\lstinline|head|}} of a \ul{singly linked list} and a value {\colorbox{CodeBackground}{\lstinline|x|}}, partition it such that all nodes less than x come before nodes greater than or equal to {\colorbox{CodeBackground}{\lstinline|x|}}. You should preserve the original relative order of the nodes in each of the two partitions.

\subsection*{Solution}
\begin{lstlisting}
ListNode* partition(ListNode* head, int x) {
	ListNode* less_dummy_head = new ListNode();
	ListNode* greater_dummy_head = new ListNode();
	ListNode* less_ptr = less_dummy_head;
	ListNode* greater_ptr = greater_dummy_head;
	while (head) {
		if (head->val < x) {
			less_ptr->next = head;
			less_ptr = less_ptr->next;
		} else {
			greater_ptr->next = head;
			greater_ptr = greater_ptr->next;
		}
		head = head->next;
	}
	less_ptr->next = greater_dummy_head->next;
	greater_ptr->next = nullptr;
	return less_dummy_head->next;
}
\end{lstlisting}

\section{LC 0083 - Remove Duplicates from Sorted List}\label{lc0083}
Given the {\colorbox{CodeBackground}{\lstinline|head|}} of a \ul{sorted singly linked list}, delete all duplicates such that each element appears only once. Return the singly linked list sorted as well.\\

Examples:
\begin{itemize}
\item {\colorbox{CodeBackground}{\lstinline|[1 -> 1 -> 2] --> [1 -> 2]|}}
\item {\colorbox{CodeBackground}{\lstinline|[1 -> 1 -> 2 -> 3 -> 3] --> [1 -> 2 -> 3]|}}
\end{itemize}

\subsection*{Solution}
\begin{lstlisting}
ListNode* deleteDuplicates(ListNode* head) {
	ListNode* curr = head;
	while (curr != nullptr && curr->next != nullptr) {
		if (curr->val == curr->next->val) {
			ListNode* to_delete = curr->next;
			curr->next = to_delete->next;
			delete to_delete;
		} else {
			curr = curr->next;
		}
	}
	return head;
}
\end{lstlisting}

\subsection*{Related}
\begin{itemize}
	\item \hyperref[lc0026]{LC 0026 - Remove Duplicates from Sorted Array}
	\item \hyperref[lc0080]{LC 0080 - Remove Duplicates from Sorted Array II}
	\item \hyperref[lc0083]{LC 0083 - Remove Duplicates from Sorted List}
	\item \hyperref[lc0082]{LC 0082 - Remove Duplicates from Sorted List II}
\end{itemize}

\section{LC 0082 - Remove Duplicates from Sorted List II}\label{lc0082}
Given the {\colorbox{CodeBackground}{\lstinline|head|}} of a \ul{sorted singly linked list}, delete all nodes that have duplicate numbers, leaving only distinct numbers from the original list. Return the singly linked list sorted as well.\\

Examples:
\begin{itemize}
\item {\colorbox{CodeBackground}{\lstinline|[1 -> 2 -> 3 -> 3 -> 4 -> 4 -> 5] --> [1 -> 2 -> 5]|}}
\item {\colorbox{CodeBackground}{\lstinline|[1 -> 1 -> 1 -> 2 -> 3] -> [2 -> 3]|}}
\end{itemize}

\subsection*{Solution}
\begin{lstlisting}
ListNode* deleteDuplicates(ListNode* head) {
	ListNode* dummy_head = new ListNode();
	dummy_head->next = head;
	ListNode* prev = dummy_head;
	while (head != nullptr) {
		if (head->next == nullptr || head->val != head->next->val) {
			prev->next = head;
			prev = prev->next;
			head = head->next;
		} else {
			int duplicateVal = head->val;
			while (head != nullptr && head->val == duplicateVal) {
				ListNode* temp = head;
				head = head->next;
				delete temp;
			}
		}
	}
	prev->next = nullptr;
	return dummy_head->next;
}
\end{lstlisting}

\subsection*{Related}
\begin{itemize}
	\item \hyperref[lc0026]{LC 0026 - Remove Duplicates from Sorted Array}
	\item \hyperref[lc0080]{LC 0080 - Remove Duplicates from Sorted Array II}
	\item \hyperref[lc0083]{LC 0083 - Remove Duplicates from Sorted List}
	\item \hyperref[lc0082]{LC 0082 - Remove Duplicates from Sorted List II}
\end{itemize}

\section{LC 0019 - Remove Nth Node From End of List}\label{lc0019}
Given the {\colorbox{CodeBackground}{\lstinline|head|}} of a \ul{singly linked list}, remove the {\colorbox{CodeBackground}{\lstinline|n|}}th node from the end of the list and return the new list.

\subsection*{Solution}
\begin{lstlisting}
ListNode* removeNthFromEnd(ListNode* head, int n) {
	ListNode* dummy_head = new ListNode();
	dummy_head->next = head;
	ListNode* p1 = dummy_head;
	ListNode* p2 = dummy_head;
	for (int i = 0; i <= n; ++i) { p2 = p2->next; }
	while (p2) {
		p2 = p2->next;
		p1 = p1->next;
	}
	ListNode* to_delete = p1->next;
	p1->next = to_delete->next;
	delete to_delete;
	return dummy_head->next;
}
\end{lstlisting}

\section{LC 0021 - Merge Two Sorted Lists}\label{lc0021}
Given the heads of two \ul{sorted singly linked lists} {\colorbox{CodeBackground}{\lstinline|list1|}} and {\colorbox{CodeBackground}{\lstinline|list2|}} in \ul{non-decreasing order}. Merge the two lists into a sorted list and return it.\\

Examples:
\begin{itemize}
	\item {\colorbox{CodeBackground}{\lstinline|list1 = [1 ->2 -> 4], list2 = [1 -> 3 -> 4] --> [1 -> 1 -> 2 -> 3 -> 4 -> 4]|}}
	\item {\colorbox{CodeBackground}{\lstinline|list1 = [], list2 = [] --> []|}}
	\item {\colorbox{CodeBackground}{\lstinline|list1 = [], list2 = [0] --> [0]|}}
\end{itemize}

\subsection*{*Solution 1 - Iterative (Extra Space)}\label{solution:lc0021_iterative1}
\begin{lstlisting}
ListNode* mergeTwoLists(ListNode* l1, ListNode* l2) {
	ListNode* dummy_head = new ListNode();
	ListNode* curr = dummy_head;
	while (l1 || l2) {
		int val1 = l1 ? l1->val : std::numeric_limits<int>::max();
		int val2 = l2 ? l2->val : std::numeric_limits<int>::max();
		if (val1 < val2) {
			curr->next = new ListNode(l1->val);
			curr = curr->next;
			l1 = l1->next;
		} else {
			curr->next = new ListNode(l2->val);
			curr = curr->next;
			l2 = l2->next;
		}
	}
	return dummy_head->next;
}
\end{lstlisting}

\subsection*{Solution 2 - Iterative (In-place)}\label{solution:lc0021_iterative2}
\begin{lstlisting}
ListNode* mergeTwoLists(ListNode* l1, ListNode* l2) {
	ListNode* dummy_head = new ListNode();
	ListNode* curr = dummy_head;
	while (l1 && l2) {
		if (l1->val < l2->val) {
			curr->next = l1;
			l1 = l1->next;
		} else {
			curr->next = l2;
			l2 = l2->next;
		}
		curr = curr->next;
	}
	curr->next = l1 ? l1 : l2;
	return dummy_head->next;
}
\end{lstlisting}

\subsection*{Other Solutions}
\begin{itemize}
\item \hyperref[solution:lc0021_recursion]{Recursion}
\end{itemize}

\subsection*{Related - Merge Sort}
\begin{itemize}
\item \hyperref[lc0088]{LC 0088 - Merge Sorted Array}
\item \hyperref[lc0021]{LC 0021 - Merge Two Sorted Lists}
\item \hyperref[lc0148]{LC 0148 - Sort List}
\end{itemize}

\section{LC 0148 - Sort List}\label{lc0148}
Given the {\colorbox{CodeBackground}{\lstinline|head|}} of a \ul{singly linked list}, return the list after sorting it in ascending order.\\

Examples:
\begin{itemize}
\item {\colorbox{CodeBackground}{\lstinline|[4 -> 2 -> 1 -> 3] --> [1 -> 2 -> 3 -> 4]|}}
\item {\colorbox{CodeBackground}{\lstinline|[-1 -> 5 -> 3 -> 4 -> 0] --> [-1 -> 0 -> 3 -> 4 -> 5]|}}
\item {\colorbox{CodeBackground}{\lstinline|[] --> []|}}
\end{itemize}

\subsection*{Solution - Merge Sort}\label{solution:lc0148_merge_sort}
Merge Sort is better than Quick Sort, because \ul{merge} is easier than \ul{partition} for a linked list.
\begin{lstlisting}
ListNode* sortList(ListNode* head) {
  if (!head || !head->next) { return head; }
  ListNode* slow = head;
  ListNode* fast = head;
  ListNode* prev_mid = nullptr;
  while (fast && fast->next) {
    prev_mid = slow;
    slow = slow->next;
    fast = fast->next->next;
  }
  prev_mid->next = nullptr;
  ListNode* l1 = sortList(head);
  ListNode* l2 = sortList(slow);
  return Merge(l1, l2);
}

ListNode* Merge(ListNode* l1, ListNode* l2) {
  ListNode* dummy_head = new ListNode(0);
  ListNode* curr = dummy_head;
  while (l1 && l2) {
    if (l1->val < l2->val) {
      curr->next = l1;
      l1 = l1->next;
    } else {
      curr->next = l2;
      l2 = l2->next;
    }
    curr = curr->next;
  }
  if (l1) {
    curr->next = l1;
  } else {
    curr->next = l2;
  }
  return dummy_head->next;
}
\end{lstlisting}

\subsection*{Other Solutions}
\begin{itemize}
%\item \hyperref[solution:lc0148_merge_sort]{Merge Sort}
\item \hyperref[solution:lc-148_priority_queue]{Priority Queue}
\end{itemize}

\subsection*{Related - Merge Sort}
\begin{itemize}
\item \hyperref[lc0088]{LC 0088 - Merge Sorted Array}
\item \hyperref[lc0021]{LC 0021 - Merge Two Sorted Lists}
\item \hyperref[lc0148]{LC 0148 - Sort List}
\end{itemize}


\section{LC 0023 - Merge k Sorted Lists}\label{lc0023}
You are given an array of {\colorbox{CodeBackground}{\lstinline|k|}} linked-lists lists, each linked-list is sorted in \ul{ascending order}. Merge all the linked-lists into one sorted linked-list and return it.\\

Example:
\begin{itemize}
	\item {\colorbox{CodeBackground}{\lstinline|lists = [[1,4,5],[1,3,4],[2,6]] --> [1,1,2,3,4,4,5,6]|}}
	\item {\colorbox{CodeBackground}{\lstinline|lists = [] --> []|}}
	\item {\colorbox{CodeBackground}{\lstinline|lists = [[]] --> []|}}
\end{itemize}

\section{LC 0138 - Copy List with Random Pointer}\label{lc0138}
A linked list of length {\colorbox{CodeBackground}{\lstinline|n|}} is given such that \ul{each node contains an additional random pointer}, which could point to any node in the list, or {\colorbox{CodeBackground}{\lstinline|nullptr|}}. \\

The {\colorbox{CodeBackground}{\lstinline|Node|}} is defined as follows:
\begin{lstlisting}
struct Node {
	Node(int _val) {
		val = _val;
		next = nullptr;
		random = nullptr;
	}
	
	int val;
	Node* next;
	Node* random;
};
\end{lstlisting}

Construct a \ul{deep copy} of the list and return it.

\subsection*{*Solution 1 - Hash Map}
\begin{lstlisting}
Node* copyRandomList(Node* head) {
	if (!head) { return nullptr; }
	std::unordered_map<Node*, Node*> origin2copy;
	Node* curr = head;
	while (curr) {
		origin2copy[curr] = new Node(curr->val);
		curr = curr->next;
	}
	for (auto& pair : origin2copy) {
		Node* origin = pair.first;
		Node* copy = pair.second;
		if (origin->next) { copy->next = origin2copy[origin->next]; }
		if (origin->random) { copy->random = origin2copy[origin->random]; }
	}
	return origin2copy[head];
}
\end{lstlisting}

\subsection*{Solution 2}
\begin{lstlisting}
Node* copyRandomList(Node* head) {
	if (!head) { return nullptr; }
	Node* curr = head;
	// first pass: make copy of each node and insert it to the original list
	while (curr) {
		Node* copy = new Node(curr->val);
		copy->next = curr->next;
		curr->next = copy;
		curr = copy->next;
	}
	// second pass: assign random pointers for the copy nodes
	curr = head;
	while (curr) {
		if (curr->random) { curr->next->random = curr->random->next; }
		curr = curr->next->next;
	}
	// third pass: extract the copy list
	curr = head;
	Node* dummy_head = new Node(0);
	Node* copy = dummy_head;
	while (curr) {
		Node* next = curr->next;
		copy->next = next;
		curr->next = next->next;
		curr = curr->next;
		copy = copy->next;
	}
	return dummy_head->next;
}
\end{lstlisting}

\subsection*{Related}
\begin{itemize}
	\item \hyperref[lc0138]{LC 0138 - Copy List with Random Pointer}
	\item \hyperref[lc0133]{LC 0133 - Clone Graph}
\end{itemize}

\section{LC 0146 - LRU Cache}
Design a data structure that follows the constraints of a \ul{Least Recently Used (LRU) cache}.\\

Implement the {\colorbox{CodeBackground}{\lstinline|LRUCache|}} class:
\begin{itemize}
	\item {\colorbox{CodeBackground}{\lstinline|LRUCache(int capacity)|}} - Initialize the LRU cache with positive size {\colorbox{CodeBackground}{\lstinline|capacity|}}.
	\item {\colorbox{CodeBackground}{\lstinline|int get(int key)|}} -  Return the value of the {\colorbox{CodeBackground}{\lstinline|key|}} if the key exists, otherwise return {\colorbox{CodeBackground}{\lstinline|-1|}}.
	\item {\colorbox{CodeBackground}{\lstinline|void put(int key, int value)|}} - Update the value of the {\colorbox{CodeBackground}{\lstinline|key|}} if the {\colorbox{CodeBackground}{\lstinline|key|}} exists. Otherwise, add the {\colorbox{CodeBackground}{\lstinline|key-value|}} pair to the cache. If the number of keys exceeds the {\colorbox{CodeBackground}{\lstinline|capacity|}} from this operation, evict the least recently used key.
\end{itemize}

The functions {\colorbox{CodeBackground}{\lstinline|get|}} and {\colorbox{CodeBackground}{\lstinline|put|}} must each run in {\colorbox{CodeBackground}{\lstinline|O(1)|}} average time complexity.

\subsection*{Solution}
\begin{lstlisting}
class LRUCache {
 public:
	LRUCache(int capacity) : capacity_(capacity) {}
	
	int get(int key) {
		auto it = cache_.find(key);
		if (it == cache_.end()) { return -1; }
		items_.splice(items_.begin(), items_, it->second);
		return it->second->second;
	}
	
	void put(int key, int value) {
		auto it = cache_.find(key);
		if (it != cache_.end()) {
			it->second->second = value;
			items_.splice(items_.begin(), items_, it->second);
			return;
		}
		if (items_.size() == capacity_) {
			auto last = items_.back();
			cache_.erase(last.first);
			items_.pop_back();
		}
		items_.emplace_front(key, value);
		cache_[key] = items_.begin();
	}
	
 private:
	int capacity_;
	std::list<std::pair<int, int>> items_;
	std::unordered_map<int, std::list<std::pair<int, int>>::iterator> cache_;
};
\end{lstlisting}